%----------------------------------------------------------------------------------------
%	PACKAGES AND OTHER DOCUMENT CONFIGURATIONS
%----------------------------------------------------------------------------------------
\documentclass[a4paper,11pt]{article}
\usepackage[a4paper,textwidth=140mm,textheight=245mm]{geometry}
\usepackage[utf8]{inputenc}
\usepackage{listings}
\usepackage{graphicx}
\usepackage{amsmath}
\usepackage{multicol}
\makeatletter
\renewcommand{\section}{\@startsection
   {section}%                         name
   {1}%                               level
   {0mm}%                             indent
   {-1.5\baselineskip}%               space above header
   {0.5\baselineskip}%                space under header
   {\sffamily\bfseries\upshape\normalsize}}% style
\renewcommand{\subsection}{\@startsection
   {subsection}%                      name
   {2}%                               level
   {0mm}%                             indent
   {-0.75\baselineskip}%              space above header
   {0.25\baselineskip}%               space under header
   {\rmfamily\normalfont\slshape\normalsize}}% style
\renewcommand{\subsubsection}{\@startsection
   {subsubsection}%                    name
   {3}%                               level
   {-10mm}%                             indent
   {-0.75\baselineskip}%              space above header
   {0.25\baselineskip}%               space under header
   {\rmfamily\normalfont\slshape\normalsize}}% style
\makeatother
\begin{document}

\begin{titlepage}
\title{TAMS27\\
Sammanfattning}
\author{Martin Söderén\\ marso329@student.liu.se\\900929-1098}
\date{\today}
\maketitle




\vfill % Fill the rest of the page with whitespace

\thispagestyle{empty}

\end{titlepage}
\begin{multicols}{2}

\section{Kontinuerlig tvådimensionell stokastiskt variabel}

\subsection{Täthetsfunktion}
En täthetsfunktions
$$f(x,y)$$
 duger om den uppfyller:
 $$f(x,y)\geq 0 \ \text{för alla x,y}$$
 $$\int_{R^2}{f(x,y)dxdy}=1$$
 
 \subsection{Marginella täthetsfunktioner}
 $$f_X(x)=\int_{-\infty}^{\infty}{f_{X,Y}(x,y)dy}$$
 $$f_Y(y)=\int_{-\infty}^{\infty}{f_{X,Y}(x,y)dx}$$
 \subsection{övrigt}
 $$P(X<Y)=\int_{0}^{\infty}{\int_{0}^{Y}{f_{X,Y}(x,y)dx dy}}$$
 
 \section{Betingad sannolikhet}
 $$P(X=j|Y=k)=\frac{P(X=j,Y=k)}{P(Y=k)}$$
$$(X=j|Y=k)=$$
$$\frac{P(Y=j|X=k)P(X=k)}{P(Y=k)}$$
$$f_{X|Y=y(x)}=\frac{f_{X,Y}(x,y)}{\int_{-\infty}^{\infty}{f_{X,Y}(t,y)dt}}$$
 \section{Räkneregler}
 $$P(A)=P(A|B)P(B)+P(A|C)P(C)$$

 
 \section{Väntevärde}
 Säger vart massan är belägen i genomsnitt
 $$E(X)=\int_{-\infty}^{\infty}xf_X(x)dx$$
 $$E(X) \ \text{betecknas ibland } \mu$$
 Vid likformig fördelning på (a,b):
 $$E(X)=(a+b)/2$$
 
 \section{Varians}
Spridningsmått som anger hur sprid täthetsfunktionen är.
$$V(X)=\int_{-\infty}^{\infty}(x-\mu)^2f_X(x)dx$$
$$E(X^2)=\int_{-\infty}^{\infty}x^2f_x{(x)dx}$$
$$V(X)=E(X^2)-(E(X))^2$$
Betecknas ibland $\sigma^2$\\
Vid likformig fördelning på (a,b):
$$V(X)=(b-a)/\sqrt{12}$$
\section{Kovarians}
$$C(X,Y)=E(XY)-E(X)E(Y)$$
$$E[XY]=\sum \sum xyP(X=x,Y=y)$$
$$E(X)=\sum \sum x*f_{X,Y}(x,y)$$

\section{korrelation}
$$\rho(X,Y)=\frac{C(X,Y)}{\sqrt{V(X)V(Y)}}$$

\section{min och max}
$$P(a*min(X,Y)>b)=P(min(X,Y)>\frac{b}{a})$$
$$=P(X>\frac{b}{a})P(Y>\frac{b}{a})$$

 \section{Oberoende stokastiska variabler}
 Två stokastiska variabler är oberoende om och endast om följande är uppfyllt:
 $$F_{X,Y}(x,y)=F_X(x)F_Y(y) \ \text{för alla x och y}$$
 
 \subsection{Summa av stokastiska variabler} Om alla variabler är oberoende och fördelade enligt $N(\mu,\sigma)$ så är summan fördelad enligt:
 $$N(\mu,\sigma/\sqrt{n})$$
 där n är antalet variabler
 OBS:
 $$P(X\leq x)=P(\frac{X-\mu}{\frac{\sigma}{ \sqrt{n}}})$$
 Om man har två variabler $X=N(x,y)$ och Y=N(a,b) och vill veta sannolikheten att $X>Y$
 så söker man $X-Y>0$. Som har väntevärde x-a och standardavvikelse $\sqrt{y^2+b^2}$
 \\
 väntevärde för en summa av stokastiska variabler:
 $$\frac{1}{n} \sum_{i=1}^{n}{X_i}$$
 \section{Centrala gränsvärdessatsen}
 Om man har en mängd stokastiska variabler
 $$X_1,X_2....X_n$$
 som är normalfördelade 
 $$N(x,y)$$
 och man skapar en stokastisk variabel av summan
 $$Y=\sum_1^iX_i$$
 så är y fördelad approximativt
 $$N(nx,ny^2)$$
 
 \section{Poisson fördelning}
 Sannolikhetsfunktionen är exponentiell. Beskrivs bäst med ett exempel:
 Anta att en intensitet följer en poissonprocess så det sker x gånger per timme. Du vill veta hur stor sannolikheten är det detta sker k gånger under y timmar.
 $$\mu=xy$$
 sannolikheten att det sker k gånger är(obs k=0 innebär att det sker en gång):
 $$p_X(k)=\frac{\mu^k}{k!}e^{-\mu}$$
 
 $$P_X(X>x)=1-P_X(X\leq x)=1-\sum_{k=0}^{x}{P_X(X=k)}$$
 
 \section{Binomial fördelning}
 sannolikhet p att A inträffar i ett enskilt försök. Om n oberoende försök utförs och X är antalet gånger som A inträffar så är X  BIN(n,p)
 
 $$Bin(\mu,\sigma)\approx Poi(\mu \sigma)$$
 
 $$Bin(\mu,\sigma)\approx N(\mu \sigma,\mu \sigma(1-\sigma))$$
  
 \section{Exponential fördelning}
Om en händelse är exponentialfördelad med $\lambda=3$ så är sannolikheten ett det sker efter x tidsenheter
$$e^{-\lambda x}$$
täthetsfunktion
$$f_X(x)=\lambda e^{-\lambda x}$$
fördelningsfunktion:
$$F_X(x)=1-e^{-\lambda x}$$
 
 \section{Normalfördelning}
 \subsection{normalisering}
 antag att man har X som är N(x,y) som ej är på normalform dvs $X\neq N(0,1)$
 och man vill veta $P(X\leq z)$
 $$P(X\leq z)=P(\frac{X-x}{\sqrt y}\leq \frac{z-x}{\sqrt y})=\Phi(\frac{z-x}{\sqrt y})$$
$$P(a<X\leq b)=\Phi{(b)}-\Phi{(a)}$$
\end{multicols}
\end{document}

