%----------------------------------------------------------------------------------------
%	PACKAGES AND OTHER DOCUMENT CONFIGURATIONS
%----------------------------------------------------------------------------------------
\documentclass[a4paper,11pt]{article}
\usepackage[a4paper,textwidth=140mm,textheight=245mm]{geometry}
\usepackage[utf8]{inputenc}
\usepackage{listings}
\usepackage{graphicx}
\usepackage{float}
\usepackage{caption}
\makeatletter
\renewcommand{\section}{\@startsection
   {section}%                         name
   {1}%                               level
   {0mm}%                             indent
   {-1.5\baselineskip}%               space above header
   {0.5\baselineskip}%                space under header
   {\sffamily\bfseries\upshape\normalsize}}% style
\renewcommand{\subsection}{\@startsection
   {subsection}%                      name
   {2}%                               level
   {0mm}%                             indent
   {-0.75\baselineskip}%              space above header
   {0.25\baselineskip}%               space under header
   {\rmfamily\normalfont\slshape\normalsize}}% style
\renewcommand{\subsubsection}{\@startsection
   {subsubsection}%                    name
   {3}%                               level
   {-10mm}%                             indent
   {-0.75\baselineskip}%              space above header
   {0.25\baselineskip}%               space under header
   {\rmfamily\normalfont\slshape\normalsize}}% style
\makeatother
\begin{document}
\begin{titlepage}
\title{TGTU49 Preparation:\\
The shock of the old questions}
\author{Martin Söderén}
\date{\today}
\maketitle



\vfill % Fill the rest of the page with whitespace

\thispagestyle{empty}

\end{titlepage}

%----------------------------------------------------------------------------------------
%	TABLE OF CONTENTS
%----------------------------------------------------------------------------------------
\section{Question 1}
For 100 years ago most new inventions that changed the world where constructed in sheds/workrooms of dedicated people. Today most inventions are created laboratories owned by multinational companies that spend huge amounts of money on the development. Is it still possible for the shedworkingman to make inventions that change the world?(page 192-)
\newline
\newline
I would consider yes it is possible, but the problem is not inventing but getting it to market. Let say I invent a revolutionary new idea and tries to get it to market. The most probably outcome is either that I can get it to market because that requires huge amount of investment. If I manage to get all the money required and starts marketing it a larger company will probably steal the idea and claim it as their own and we will get into a legal process which they will win because they have the money to spend on lawyers and they can go on forever.
One outcome might be that I contact a large company and asks if they want to invest in the idea, they will probably take most of the profits from it. 

\section{Question 2}
Has the invention of cheap massproducable firearms enabled us to kill more effectively?\newline
\newline
I would consider no, throughout the history people has always find ways to killing them self. Take for hutu versus tutsies. They managed to kill of most hutu(800 000) in four months using only machetes. Let say that the WW2 would have been fought using only machetes, my guess is that the same amount of people would have died. Their tactics would be different but the killing would be the same. 

\section{Question 3}
In today’s society we are surrounded by computer all the time, we have access to most of the information in the world just a few clicks away. How come we still use medieval technology such as papers, paper-clips and post-it to share information with each other?
\newline
\newline
I think that a paperless society is a possibility. However it is easier to write down ideas on a post-it and put it on the screen. This could however be exchanged for having virtual post-its on the desktop. It might be such as Edgerton puts it, they are a little bit better than the alternative. We could definitely switch to putting virtual post-its on the desktop but that would require us to start the computer when we want to read. We have the possibility that the computer one day will crash without being able to recover our precious notes.(Page 8)


\end{document}