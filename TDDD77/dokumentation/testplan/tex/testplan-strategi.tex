\section{Strategier}
För att kunna utföra alla tester krävs det någon form utav strategi. Följande delar har vi valt.

\subsection{Enhetstester}
För enheter och funktioner använder vi oss utav ''Black Box Testing'', det vill säga att vi ger indata till en funktion och kollar ifall utdatan är samma som den väntade. Testen som ska utföras är härledda från kravspecifikationen, projektplanen, kavalitetsplanen och architekturdokumentet.

\subsection{Kompabilitetstester}
Kunden har specificerat att programmet ska kunna köras på flera olika operativsystem. Därför krävs det att alla test körs på alla plattformar för garantera kompabiliteten.

\subsection{Integrationstester}
När mindre delar kod är klarskrivna och testade ska de kopplas ihop. Då krävs tester för att se till att alla funtioner och metoder kommunicerar med varandra på rätt sätt. Dessa tester utförs på samma sätt som enhetstesterna men inriktar sig mer på gränssnitten mellan moduler.

\subsection{Systemtester}
När underliggande funktion är implementerad och färdigtestad måste test av hela systemet utföras. Detta för att garantera prestandan hos QuadOpt och för att se till att alla generella krav är uppfyllda och att alla features finns med. 
