\section{MATLAB}
Beskrivning för installation och användande av QuadOpt i MATLAB.

\subsection{Installera QuadOpt i MATLAB}
\begin{enumerate}
	\item Starta MATLAB och ställ dig i \emph{/matlab} på valfritt sätt. 
	\item Skriv \textbf{build} i MATLAB-terminalen.
	\item Klart. QuadOpt är nu redo att använda i MATLAB.
	\newline
	\newline
	\textbf{OBS!} Kom ihåg att en C-kompilator måste vara installerad för att detta ska fungera!
\end{enumerate}

\subsection{Användning}
Anropa QuadOpt lösaren i MATLAB genom att skriva

\begin{lstlisting}
z = quadopt(Q, q, E, h, F, g, iter, time);
\end{lstlisting}

\begin{itemize}
	\item z - Matris där lösningen sparas.
	\item Q - Matris innehållande den kvadratiska delen av problemet.
	\item q - Matris innehållande den linjära delen av problemet.
	\item E - Ekvivalensbivillkorsmatrisen.
	\item h - Ekvivalensbivillkoren i högerledet.
	\item F - Olikhetsbivillkorsmatrisen (\textbf{OBS!} det är större än eller lika med g).
	\item g - Olikhetsbivillkoren i högerledet.
	\item iter - Max antal iterationer innan lösaren avbryter. Input som 0 anger oändligt många iterationer och input som 1 ger en iteration osv. 
	\item time - Max tid i mikrosekunder innan lösaren avbryter. Input som 0 anger oändligt många mikrosekunder.
	\item \textbf{OBS!} Om någon matris inte finns till användning kan \texttt{[]} användas som input. Ett exempel är att om q matrisen inte finns, så kan indatan stå som \texttt{[]}. Ett exempel på ett quadopt-anrop finns i bilaga C tillsammans med ett exempel.
\end{itemize}
