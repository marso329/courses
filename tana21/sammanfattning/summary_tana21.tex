%----------------------------------------------------------------------------------------
%	PACKAGES AND OTHER DOCUMENT CONFIGURATIONS
%----------------------------------------------------------------------------------------
\documentclass[a4paper,11pt]{article}
\usepackage[a4paper,textwidth=140mm,textheight=245mm]{geometry}
\usepackage[utf8]{inputenc}
\usepackage{listings}
\usepackage{graphicx}
\usepackage{amsmath}
\usepackage{mathtools}
\usepackage{subscript}
\usepackage{tikz}
\usepackage{float}
\usepackage[]{algorithm2e}
\makeatletter
\renewcommand{\section}{\@startsection
   {section}%                         name
   {1}%                               level
   {0mm}%                             indent
   {-1.5\baselineskip}%               space above header
   {0.5\baselineskip}%                space under header
   {\sffamily\bfseries\upshape\normalsize}}% style
\renewcommand{\subsection}{\@startsection
   {subsection}%                      name
   {2}%                               level
   {0mm}%                             indent
   {-0.75\baselineskip}%              space above header
   {0.25\baselineskip}%               space under header
   {\rmfamily\normalfont\slshape\normalsize}}% style
\renewcommand{\subsubsection}{\@startsection
   {subsubsection}%                    name
   {3}%                               level
   {-10mm}%                             indent
   {-0.75\baselineskip}%              space above header
   {0.25\baselineskip}%               space under header
   {\rmfamily\normalfont\slshape\normalsize}}% style
\makeatother
\begin{document}

\begin{titlepage}
\title{Tana21 Summary:}
\author{Martin Söderén\\ marso329@student.liu.se\\900929-1098}
\date{\today}
\maketitle
\vfill % Fill the rest of the page with whitespace
\thispagestyle{empty}
\end{titlepage}
\section{Relativa fel}
$\dfrac{\Delta a}{a}$ där $\Delta a=\bar{a}-a$ där $\bar{a}$ är det absoluta felet.
\subsection{Exempel}
Närmevärdet 150824 har ett relativt fel på 1.2. Hur många signifikanta siffror har närmevärdet? 
$$\dfrac{\Delta a}{150824}=1.2$$
$$\Delta a=180988.8$$ 
$$\bar{a}=\Delta a +a=180988.8+150824=331812.8$$
Vilket är mycket större än närmevärdet så det har inga signifikanta siffror.

\section{Normaliserad form i ett flyttalssystem}
\subsection{Exempel}
Skriv 150824 i flyttalssystem med två decimaler, basen 10 och exponenet $\pm$ 9. Svar:
$1.51*10^5$

\section{Normer}
\subsection{Vektor}
x är en vektor
$$||x||_\infty=max_i |x_i|$$ 
$$||x||_p=(\sum_i|x_i|^p)^{1/p}$$
\subsection{Matriser}
Beräkna:
$$\dfrac{||\Delta x||_{\infty}}{||x||_{\infty}}$$
$$\dfrac{||\Delta x||_{\infty}}{||x||_{\infty}}=||A||_{\infty}*||A^{-1}||_{\infty}*\dfrac{||\Delta b||_{\infty}}{||b||_{\infty}}$$
Där
$||K||_{\infty}$ är max summan för alla radvektorer för en godtycklig matris \textit{K}.

\section{Trapetsregeln}
Används för integrering

\section{Trapetsmetoden}
Används för att lösa differentialsekvationer av första ordningen.

\section{Skriv om till ett system av första ordningen}
\subsection{Exempel}
$$y''=2y'-y/x,y(1)=4,y'(1)=1$$
Sätt
$$u=y,v=y'$$
Vilket ger:
$$u'=v, v'=2v-u/x,u(1)=4,v(1)=1$$	
$$$$

\section{Bestämma ett närmevärde och totalla felet till en funktion}
\subsection{Exempel}
$$f(r,\alpha)=\dfrac{r^2}{2}(\alpha-sin(\alpha))$$
$$r =1.50\pm 0.03$$
dvs $\Delta r=0.03$
$$\alpha=0.22\pm0.01$$
och $\Delta \alpha=0.01$
\newline
Maximalfeluppskattning:
$$|\Delta  y|=\sum_{k=1}^{n}|\dfrac{\delta f}{\delta x_k}(\bar{x_1},...\bar{x_n})\Delta x_n|$$
Dvs maximalfeluppskattning är summan av funktionen funktionens partiella derivator gånger variabelns absoluta fel.
\newline
Detta ger
$$|\Delta y|=|1.5(0.22-sin(0.22))0.03|+|\dfrac{1.5^2}{2}(1-cos(0.22))0.01|\approx 3.51*10^{-4}$$
$\bar{y}=0.00199167$ och med $\bar{y}=0.00199$ blir $|R_b|$(avrundningsfelet) $< 0.5*10^{-5}$ 
Totala felet blir $0.5*10^{-5}+3.51*10^{-4}\text{ vilket är }<3.6*10^{-4}$
Så $y=0.00199\pm 0.00036$

\section{Minsta kvadratmetoden}
Vi har en massa mätpunkter som vi vill skapa en approximativ funktion till.
\begin{table}[H]
  \centering
  \begin{tabular}{|l|l|l|l|}
    \hline
    $x_1$ & $x_2$  & ...  & $x_n$ \\ \hline
    $y_1$ &  $y_2$ &  & $y_n$ \\ \hline

  \end{tabular}
  \caption{Mätpunkter}
  \label{tab:funktion2}
\end{table}
\subsection{Exempel för första gradens polynom}
vi vill lösa:
$$a+bx_1=y_1$$
$$a+bx_2=y_2$$
\centering
...
$$a+bx_n=y_n$$
Så bra som möjligt. Sätt upp det i matrisform
\[ \left( \begin{array}{ccc}
1 & x_1  \\
1 & x_2  \\
\vdots & \vdots  \\
1 & x_n  \end{array} \right)
 \left( \begin{array}{ccc}
a   \\
b     \end{array} \right)=
\left( \begin{array}{ccc}
 y_1  \\
 y_2  \\
 \vdots  \\
 y_n  \end{array} \right)
\]
$Ax=b$
Beräkna:
$$A^tAx=A^tb$$
och lös ekvationssystemet för att få fram a och b.

\section{Lagrange interpolation}
$p_j(x)=y_j\prod_{k=1, k\neq j}\dfrac{x-x_k}{x_j-x_k}$

\end{document}