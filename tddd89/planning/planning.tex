%----------------------------------------------------------------------------------------
%	PACKAGES AND OTHER DOCUMENT CONFIGURATIONS
%----------------------------------------------------------------------------------------
\documentclass[a4paper,11pt]{article}
\usepackage[a4paper,textwidth=140mm,textheight=245mm]{geometry}
\usepackage[utf8]{inputenc}
\usepackage{listings}
\usepackage{graphicx}
\usepackage{mathtools}
\usepackage{subscript}
\usepackage{tikz}
\usepackage{float}
\usepackage[]{algorithm2e}
\makeatletter
\renewcommand{\section}{\@startsection
   {section}%                         name
   {1}%                               level
   {0mm}%                             indent
   {-1.5\baselineskip}%               space above header
   {0.5\baselineskip}%                space under header
   {\sffamily\bfseries\upshape\normalsize}}% style
\renewcommand{\subsection}{\@startsection
   {subsection}%                      name
   {2}%                               level
   {0mm}%                             indent
   {-0.75\baselineskip}%              space above header
   {0.25\baselineskip}%               space under header
   {\rmfamily\normalfont\slshape\normalsize}}% style
\renewcommand{\subsubsection}{\@startsection
   {subsubsection}%                    name
   {3}%                               level
   {-10mm}%                             indent
   {-0.75\baselineskip}%              space above header
   {0.25\baselineskip}%               space under header
   {\rmfamily\normalfont\slshape\normalsize}}% style
\makeatother
\begin{document}

\begin{titlepage}
\title{TDDD89 Planning Report:}
\author{Martin Söderén\\ marso329@student.liu.se\\900929-1098}
\date{\today}
\maketitle
\vfill % Fill the rest of the page with whitespace
\thispagestyle{empty}
\end{titlepage}
\section{Preliminary title}
Real time visulization of MPI utilization in the C++ MPI library 

\section{Problem description}
MPI(Message Passing Interface) is a standardized interface for message-passing between computers and is mostly used for distributed computing on large parallel computers. There is a need for a real time platform independent visualization tool that can be used to analyze how well the computing is being performed. There exists tools that do this post-mortem such as mpe and VAMPIRE. Some tools that do this in real time but are not up to date such as XMPI.
\newline
\newline
The expected result should be base framework with some limited functionality and documentation on how to further develop the functionality since development of a complete tool will probably take more time that can be spend during this thesis.

\section{Approach}
The first phase will be to read up on how the MPI is used and read all of the relevant documentation that exists.
\newline
\newline
The second phase will be test and use the current available softwares to see how they are used and their downsides and upsides.
\newline
\newline
The third phase will be to read relevant literature of academic nature on how a optimized implementation should be implemented.
\newline
\newline
The forth phase will be to start implementing
\newline
\newline
During this whole time there will be time spent on the actual thesis. 

\section{Literature base }
\begin{itemize}
\item Using MPI - 2nd Edition: Portable Parallel Programming with the Message Passing Interface (Scientific and Engineering Computation)
\item http://www.mcs.anl.gov/research/projects/mpi/usingmpi2/
\item http://www.cs.usfca.edu/~peter/ppmpi/
\item http://moss.csc.ncsu.edu/~mueller/cluster/mpi.guide.pdf
\item http://dl.acm.org/citation.cfm?id=898758
\item https://www.vampir.eu/
\item http://www.mcs.anl.gov/research/projects/perfvis/software/MPE/
\item https://wiki.mpich.org/mpich/index.php/MPE\_by\_example
\end{itemize}

\section{Time plan}
\subsection{Pre-study}

\begin{table}[H]
  \centering
  \begin{tabular}{|l|l|l|l|l|}
    \hline
    \textbf{Activity} & Duration & Finished & Dependencies & Comment  \\ \hline
    Planning & One day & 6-11-15 & None & This document \\ \hline
    Create all latex templates & Couple of hours & 13-11-15 & None & Setup reference \\ \hline
    Reading MPI documentation & One week & 20-11-15 & None & Also testing \\ \hline
    Testing MPI tools & Three days & 25-11-15 & MPI doc & Document during \\ \hline
	 Reading academic literature & Two weeks & 9-12-15 & None & Document during \\ \hline
   Thesis draft & Three days & 12-12-15 & academic lit. & None \\ \hline
  \end{tabular}
  \caption{Pre-study time plan }
  \label{tab:funktion1}
\end{table}
\subsection{Experimental}
\begin{table}[H]
  \centering
  \begin{tabular}{|l|l|l|l|l|}
    \hline
    \textbf{Activity} & Duration & Finished & Dependencies & Comment  \\ \hline
    Implementation & Six weeks & 6-02-16 & Pre-study & None \\ \hline

  \end{tabular}
  \caption{Implementation time plan }
  \label{tab:funktion1}
\end{table}


\subsection{Writing}

\begin{table}[H]
  \centering
  \begin{tabular}{|l|l|l|l|l|}
    \hline
    \textbf{Activity} & Duration & Finished & Dependencies & Comment  \\ \hline
    Writing thesis & Two months & 6-4-16 & Implementation & None \\ \hline

  \end{tabular}
  \caption{Writing time plan }
  \label{tab:funktion1}
\end{table}

\section{Relevant courses}
\begin{itemize}
\item TDDB68	Processprogrammering och operativsystem
\item TDDD20	Konstruktion och analys av algoritmer
\item TDDD56	Multicore- och GPU-Programmering
\item TDDD89	Vetenskaplig metoder
\item TDDD25 	Distribuerade system
\item TDDD38 	Avancerad programmering i C++
\item TDDC78 	Programmering av parallelldatorer - metoder och verktyg
\end{itemize}

\end{document}