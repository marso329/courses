\section{Bakgrund}
Det svåraste problemet jag hade innan jag började med projeketet var att på ett effektivt sätt interagera med datorns filsystem. Detta löstes rätt enkelt när jag upptäckte Boost::Filesystem. Detta bibliotek är inte speciellt använt så det finns inte så mycket hjälp att hitta på forum men dokumentation för det är väldigt bra så när jag väl kom in i det så fungera det väldigt bra. Funktioner såsom att byta namn och radera filer fanns redan implementerade så detta underlättade mycket.
\\
\\
Ett annat problem som var svårare än vad jag trodde var att visa filnamn i en 3D miljö. Den enklaste lösning skulle varit att använda något bibliotek för att generera texturer med en font till exempel FreeType. Jag försökte detta och jag kommer inte ihåg varför men jag fick det inte att fungera. För att få detta snyggt så skulle texturerna behöva vara transparenta så jag skulle behöva ta hänsynt till ordningen som alla objekt renderas. Min lösning på problemet var att generera .obj filer för alla tänkbara bokstäver och tecken som kan förekomma i ett filnamn. Jag valde alla bokstäver a-z och A-Z, alla siffror samt tecknen . - \_. Detta gjordes med ett python script i cad-programmet FreeCad. För att minimera inläsningstiden så läses alla dessa modeller in vid starten av programmet och återanvänds hela tiden. För att undvika problem som kan uppkomma vid långa filnamn som sträcker sig över hela miljön så visas bara de 7 första tecknen i filnamnet.
\\
\\
Anledningen till att C++ valdes var mest för dess datastrukturer såsom Vector, Map och string. Allting som kan göras i C++ kan självklart göras i C men där så är allting klart. Detta gjorde så att jag kunde göra en väldigt allmän klass som representerar alla objekt i 3D världen. Programmet arbetar också mycket med strängar och då är String väldigt mycket smidigare än en char array.
\\
\\
Ett problem som upptäcktes senare var att när man går in i en mapp som innehåller många filer, speciellt bilder, så tar det rätt lång tid innan hela världen har genererats. Detta skulle kunna lösas genom att ha en separat tråd som förbereder alla undermappar i den nuvarande mappen. Detta har dock inte implementeras men skulle jag fortsätta utveckla TMBTRF så skulle detta vara ett av de första problemen jag skulle lösa. 
